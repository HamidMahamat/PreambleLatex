%%%%%%% Commandes Perso %%%%%%%%
\DeclarePairedDelimiter\floor{\lfloor}{\rfloor}
\DeclareMathOperator{\Ham}{\mathcal{H}}	% Opérateur hamiltonien
\DeclareMathOperator*{\argmin}{\text{argmin}}	% Opérateur argument du min
\DeclareMathOperator*{\argmax}{\text{argmax}}	% Opérateur argument du max
\DeclareMathOperator{\sinc}{\text{sinc}} % sinus cardinal
\renewcommand{\Re}{\text{Re}}			 % Partie réelle
\DeclareMathOperator{\sgn}{\text{sgn}}	 % Fonction signe
\DeclareMathOperator{\e}{\text{e}}		 % L'opérateur exponentiel e
\newcommand{\iu}{\mathrm{i}}			 % i imaginaire complexe
\newcommand{\deriv}{\text{d}}			 % le d de la différenciabilité
\newcommand\smallO{
	\mathchoice
	{{\scriptstyle\mathcal{O}}}% \displaystyle
	{{\scriptstyle\mathcal{O}}}% \textstyle
	{{\scriptscriptstyle\mathcal{O}}}% \scriptstyle
	{\scalebox{.7}{$\scriptscriptstyle\mathcal{O}$}}%\scriptscriptstyle
} % Petit O
\newcommand{\ens}[1]{\mathbb{#1}}              					%Ensemble usuelle : N, Z, R, C, Q, K
\newcommand{\abs}[1]{\left\vert {#1}\right\vert}				%Valeur absolue
\newcommand{\partiel}[2]{\frac{\partial {#1}}{\partial {#2}}}	%derivée partielle
\newcommand{\pesp}{\,}											%petit espace
\newcommand{\gesp}{\;}											%grand espace
\newcommand{\nesp}{\˽}											%normal espace
\newcommand{\grandm}{\displaystyle}								%grande écriture maths
\newcommand{\petitm}{\scriptstyle}								%petite écriture maths
\newcommand{\im}[1]{\text{Im}\,{#1}}							%Ensemble image
\newcommand{\Ker}[1]{\text{Ker}\,{#1}}							%Noyau ou Ker
\newcommand{\somme}[2]{\overset{#2}{\underset{#1}{\sum}} \,}	%Operateur de sommation
%\newcommand{\norm}[1]{\left\Vert #1\right\Vert}
\newcommand{\norm}[1]{\left\|{#1}\right\|}						%Norme vectorielle avec 2 barres
\newcommand{\fonctnorm}[1]{{\left\vert\kern-0.25ex\left\vert\kern-0.25ex\left\vert #1 % Norme fonctionnelle avec 3 barres
		\right\vert\kern-0.25ex\right\vert\kern-0.25ex\right\vert}}
\newcommand{\ensvide}{\varnothing}								%Ensemble vide
\newcommand{\limite}[1]{\lim\limits_{#1}}						%Operateur de limite
\newcommand{\boule}[2]{\mathcal{B}\left({#1},{#2}\right)}		%Boule avec centre et rayon
\newcommand{\boulef}[2]{\overline{\mathcal{B}}\left({#1},{#2}\right)} %Boule fermée 
\newcommand{\sphere}[2]{\mathcal{S}\left({#1},{#2}\right)} 				%Sphere avec centre et rayon
\newcommand{\spheref}[2]{\overline{\mathcal{S}}\left({#1},{#2}\right)} 
\newcommand{\disqf}[2]{\overline{\mathcal{D}}\left({#1},{#2}\right)}	%Disque fermée
\newcommand{\disq}[2]{\mathcal{D}\left({#1},{#2}\right)}				%Disque ouvert
\newcommand{\dif}[2]{\mathrm{d}_{#2}{#1}}								%Différentielle d'une fonction en un point
\newcommand{\Vecteur}[1]{\boldsymbol{\mathbf{#1}}}						%Vecteur gras
\newcommand{\jac}[2]{J_{#2} #1}											%Jacobienne d'une fonction en un point
\newcommand{\transp}[1]{{}^\text{t}\! #1}								%La transposée
\newcommand{\prdtscal}[2]{\langle #1, #2\rangle} 						%Produit scalaire
\newcommand{\flux}[2]{\Phi_{{#1}\rightarrow{#2}}}						%Flux d'un champs le long d'une surface
\newcommand{\ensemble}[2]{\left\lbrace   #1\; \left|\; #2 \right\rbrace  \right.} %Ensemble en compréhension
\renewcommand*{\overrightarrow}[1]{\vbox{\halign{##\cr 
			\tiny\rightarrowfill\cr\noalign{\nointerlineskip\vskip1pt} 
			$#1\mskip2mu$\cr}}}
\newcommand{\vecteur}[1]{\overrightarrow{#1}\hspace*{-2pt}}							%Vecteur avec une grande fleche
\newcommand{\Int}[4]{\int_{#1}^{#2} {#3} \mathrm{d}#4}					%Intégrale
\renewcommand{\thefootnote}{(\alph{footnote})}	
\DeclareMathOperator{\rot}{\vecteur{\text{rot}}} 						% Rotationnel
\DeclareMathOperator{\divergence}{\mathrm{div}}						% Divergence
\DeclareMathOperator{\gradient}{\vecteur{\mathrm{grad}}}				% Gradient

%%%%%% Definition de la gaussienne pour PGF %%%%%
\pgfmathdeclarefunction{gauss}{2}{%
	\pgfmathparse{1/(#2*sqrt(2*pi))*exp(-((x-#1)^2)/(2*#2^2))}%
}
%% variable de gauss(<moyenne>,<ecart-type>)


\pgfplotsset{
	legend entry/.initial=,
	every axis plot post/.code={%
		\pgfkeysgetvalue{/pgfplots/legend entry}\tempValue
		\ifx\tempValue\empty
		\pgfkeysalso{/pgfplots/forget plot}%
		\else
		\expandafter\addlegendentry\expandafter{\tempValue}%
		\fi
	},
}


%\setBold % Active la police grasse 

%%%%%%%%% Environnement Perso %%%%%%%%
\theoremstyle{definition}	 %Garde la même police pour l'environnement 
\newtheorem{question}{Question}[] % Création de l'environnement exo dont le numérotage est indépendant des autres environnement
\newtheorem{contrainte}{Contrainte}[] % Création de l'environnement question dont le numerotage depend de l'environnement exo
\pagestyle{plain} 

