%%%%%%%%%%%% Package %%%%%%%%%%%
\usepackage{xfakebold} %Pour le gras
\usepackage{esvect}
\usepackage[utf8x]{inputenc}
\usepackage[french]{babel}
\usepackage[autolanguage]{numprint} % Pour écrire les nombres comme dans la langue du text
\usepackage[T1]{fontenc}
\usepackage{amsmath,amssymb,amsfonts, amsthm}
\usepackage{systeme} 
\usepackage{empheq,etoolbox}
\usepackage{array}
\usepackage{enumitem}
\usepackage{mathrsfs}
\usepackage{mathtools}
\usepackage{cancel}
\usepackage{wrapfig}
\usepackage{stmaryrd}
\usepackage{bbm}
\usepackage{booktabs}
% Interligne
\usepackage{setspace}
\usepackage{verbatim}
% \singlespacing
%\onehalfspacing
% \doublespacing
\usepackage{multirow, multicol}
\usepackage{adjustbox}
\usepackage{cite}
\usepackage{titlesec}

\usepackage{listings}
\usepackage[framed, numbered]{matlab-prettifier}




%%%%%% Geometry %%%%%%%
\usepackage{geometry}
\geometry{left=2cm,right=2cm,top=2cm,bottom=2cm}
\savegeometry{1}

\geometry{left=1cm,right=1cm,top=2.7cm,bottom=1.4cm}
\savegeometry{2}

\geometry{left=2.5cm,right=2.5cm,top=3cm,bottom=3cm}
\savegeometry{3}

%%%%%%%%%%%%%%%%%%%%%%%%

\usepackage{graphicx,mwe,fancyhdr}
\usepackage{fancybox}
\usepackage{siunitx}
\sisetup{locale = FR,
	detect-all,range-phrase=-,     
	range-units=single,
	separate-uncertainty = true,
	multi-part-units = single
}
\usepackage{lmodern}
\usepackage{hyperref}
\hypersetup{colorlinks=true}
%\hypersetup{final}
\usepackage{tabularx}
\usepackage{color,xcolor}
\definecolor{dkgreen}{rgb}{0,0.6,0}
\definecolor{gray}{rgb}{0.5,0.5,0.5}
\definecolor{mauve}{rgb}{0.58,0,0.82}



%%%%% Graphique %%%%%%%%
%\usepackage{tikz
	\usepackage{tikz-network}       % Pour la théorie des graphes (vertices & edges)
	
	\tikzset{>=stealth}
	\usepackage{pgfplots,pgfplotstable}
	%\usepackage[miktex]{gnuplottex}
	\pgfkeys{/pgf/number format/1000 sep={\,}, /pgf/number format/use comma,  } % Pour avoir des virgules au lieu des points comme séparateur décimal dans les Axes graph
	%\pgfkeys{/pgf/plot/gnuplot call='C:\Program Files\gnuplot\bin\gnuplot'}
	\pgfplotsset{compat=newest}
	\usepackage{mathrsfs}
	\usetikzlibrary{shapes.geometric, shapes.misc, intersections,calc, arrows}
	\lstset{ %
		language=R,                     % the language of the code
		basicstyle=\footnotesize,       % the size of the fonts that are used for the code
		numbers=left,                   % where to put the line-numbers
		numberstyle=\tiny\color{gray},  % the style that is used for the line-numbers
		stepnumber=1,                   % the step between two line-numbers. If it's 1, each line
		% will be numbered
		numbersep=5pt,                  % how far the line-numbers are from the code
		backgroundcolor=\color{white},  % choose the background color. You must add \usepackage{color}
		showspaces=false,               % show spaces adding particular underscores
		showstringspaces=false,         % underline spaces within strings
		showtabs=false,                 % show tabs within strings adding particular underscores
		frame=single,                   % adds a frame around the code
		rulecolor=\color{black},        % if not set, the frame-color may be changed on line-breaks within not-black text (e.g. commens (green here))
		tabsize=2,                      % sets default tabsize to 2 spaces
		captionpos=b,                   % sets the caption-position to bottom
		breaklines=true,                % sets automatic line breaking
		breakatwhitespace=false,        % sets if automatic breaks should only happen at whitespace
		title=\lstname,                 % show the filename of files included with \lstinputlisting;
		% also try caption instead of title
		keywordstyle=\color{blue},      % keyword style
		commentstyle=\color{dkgreen},   % comment style
		stringstyle=\color{mauve},      % string literal style
		escapeinside={\%*}{*)},         % if you want to add a comment within your code
		morekeywords={*,...}            % if you want to add more keywords to the set
	} 